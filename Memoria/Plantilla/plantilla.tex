\documentclass{DeustoFDP}

\usepackage{hologo} % Paquete no necesario. Borrar en la memoria final al sustituir el texto

\hypersetup{
  pdfauthor={Cruz Enrique Borges Hern\'andez},
  pdftitle={Documentaci\'on de la plantilla LaTeX para los proyectos fin de grado
            de la Facultad de Ingenier\'ia de la Universidad de Deuto},
}

\bibliography{bib}

\begin{document}

\frontmatter
\pagestyle{plain}

% Las siguientes lineas (21--26) se pueden eliminar del documento final.
% Notese que en ese caso es necesario descomentar la linea 28 para que las
% paginas esten correctamente numeradas.
\begin{titlepage}
  \newgeometry{left=0cm,right=0cm,bottom=0cm,top=0cm}
  \includegraphics{fig/portada}
  \restoregeometry
\end{titlepage}
\cleardoublepage

%\setcounter{page}{3}

\chapter*{Resumen}

Documentaci\'on de la plantilla \hologo{LaTeX} para los Proyectos Fin de Grado
de la Universidad de Deusto. En este cap\'itulo se debe introducir un resumen de
entre 200 y 250 palabras que identifique claramente los objetivos del proyecto.
Adem\'as deber\'a contener entre 2 y 5 palabras claves separadas por comas.
N\'otese que las dos p\'aginas anteriores (p\'agina de portada) se incluyen por
completitud pero no deben ser usadas en la impresi\'on del documento.
Para usarlas, es necesario copiar la portada correspondiente en la carpeta
\verb+portadas+ a la carpeta \verb+fig+ y ponerle el nombre
\verb+portada.pdf+. Si no se desea poner la portada en el documento
\hologo{LaTeX} v\'ease los comentarios de este documento. N\'otese que en
todo caso el \'indice debe comenzar siempre en la p\'agina \textit{v}.

\vspace{2em}

{\Large\bfseries\sectionfont Descriptores}
\vspace{3\medskipamount}

Plantilla, Proyecto Fin de Grado, \hologo{LaTeXe}.

\cleardoublepage\tableofcontents
\cleardoublepage\listoffigures
\cleardoublepage\listoftables
\cleardoublepage\listoflistings

\mainmatter
\pagestyle{phdthesis}

\chapter{Introducci\'on}\label{cha:introduccion}

Este documento contiene una plantilla para la presentaci\'on de los Proyectos
Fin de Grado de la Faculta de Ingenier\'ia de la Universidad de Deusto. La
plantilla se basa en tecnolog\'ias modernas y por lo tanto necesita una
distribuci\'on de \hologo{LaTeXe} reciente (\hologo{MiKTeX} 2.9~\cite{miktex} o
TeXLive 2012~\cite{texlive}). En particular usamos:
\begin{description}
  \item[\hologo{XeLaTeX} o \hologo{LuaLaTeX}:] Debido a que en esta memoria se
    usan las fuentes Arial y Bookman Old Style que noest\'an incluidas generalmente en
    las distribuciones \hologo{LaTeXe} es necesario recurrir a compiladores que
    soporte fuentes TTF o OTF como \hologo{XeLaTeX} o \hologo{LuaLaTeX}.
    Estos compiladores solo est\'an presentes en las versiones m\'as modernas de
    \hologo{MiKTeX} (2.9) y TexLive (2012).

    La fuente Arial viene por defecto en el sistema operativo Windows\textregistered\
    mientras que en sistemas *nix se puede encontrar en el paquete
    \texttt{corefonts}~\cite{corefonts}. En los sistemas operativos derivados
    de Debian puede encontrarse en el paquete \texttt{ttf-mscorefonts-installer}.

    Por otro lado, la fuente Bookman Old Style se puede encontrar en equipos con
    el sistema operativo Windows al instalar la \emph{suit} ofim\'atica
    Office\textregistered. En entornos *nix se puede descargar
    de~\cite{fuente}\footnote{Obviamente, tambi\'en se puede instalar
    en entornos Windows\textregistered.}. La licencia nos permite redistribuir
    la fuente y por ello se encuentra en la carpeta \texttt{TeX-Gyre-Bonum}.
    Para instalarlas simplemente hay que copiarlas al directorio
    \verb+~/.fonts+ cre\'andolo si es necesario. N\'otese que \textasciitilde\ representa el
    directorio del usuario (normalmente \verb+/home/"username"+ y que el directorio
    \verb+.fonts+ empieza por un punto y por lo tanto es un directorio oculto.

  \item[\textsc{Bib}\hologo{LaTeX}:] En la actualidad es muy dif\'icil generar
    bibliograf\'ias de calidad sin usar un gestor de registros bibliogr\'aficos.
    En esta plantilla se decidido usar el moderno gestor referencias
    bibliogr\'aficas \textsc{Bib}\hologo{LaTeX}. Para su uso es necesario instalar
    tanto el paquete \texttt{biblatex} como el programa \hologo{biber}.
    La versi\'on mas reciente de \hologo{MiKTeX} (2.9) permite usar sin problemas
    estos paquetes. En entornos *nix es necesario instalar estos programas desde
    el gestor de paquetes.

    \textsc{Bib}\hologo{LaTeX} es compatible con \hologo{BibTeX} as\'i como con los
    principales gestores bibliogr\'aficos co\-mer\-cia\-les como
    RefWorks\textregistered~\cite{refworks} o
    Mendeley\textregistered~\cite{mendeley} as\'i como con los formatos
    de bases de datos bibligr\'aficas RIS, Endnote\texttrademark\ XML and
    Zotero RDFXML.

    Para m\'as informaci\'on sobre como hacer bibliograf\'ias con
    \textsc{Bib}\hologo{LaTeX} se puede consultar los diversos tutoriales que hay en la
    web, como por ejemplo el siguiente video \cite{youtube} de YouTube\textregistered,
    el wikilibro~\cite{wikilibro} o este mismo documento.

  \item[Pygments:] Los proyectos fin de grado de esta facultad contienen
    en muchas ocasiones c\'odigo fuente o pseudoc\'odigos. La representaci\'on
    de c\'odigo fuente es un tema complejo pues se necesita \enquote{escapar}
    el texto a formatear del propio c\'odigo \hologo{LaTeX}. En esta plantilla
    hemos decidido usar programa \texttt{pygments} (de la misma manera que con
    la bibliograf\'ia) para realizar esta tarea.  Sin embargo, en este caso no
    es necesario realizar una llamada externa como en el caso de \hologo{BibTeX}
    pues se la llamada se realiza al compilar el propio documento. Esto presenta
    un problema de seguridad en general y es por ello que es necesario introducir
    el par\'ametro \verb+-shell-escape+ en la llamada al compilador para permitir
    este comportamiento. Cons\'ultese los manuales del IDE en uso para m\'as
    informaci\'on sobre este proceso.

    Para su uso es necesario instalar un int\'erprete Python as\'i como el
    programa \texttt{pygments} y el paquete \texttt{minted}. En entornos *nix
    no deber\'ia de haber ning\'un problema pues estas aplicaciones est\'an
    en los repositorios de paquetes. En los sistemas derivados de Debian se puede
    instalar de la siguiente forma:
    \begin{enumerate}
      \item Instalar los paquetes: \texttt{python-setuptools} y \texttt{texlive-latex-extra}
      \item Ejecutar el siguiente comando desde una terminal: \texttt{sudo easy\_install pygments}
    \end{enumerate}

    Para su instalaci\'on en entornos Windows\texttrademark\ se puede seguir la
    siguiente gu\'ia~\cite{pygments}.

  \item[\texttt{Hyperref}:] Dado que uno de los objetivos de las plantillas es generar
    documentos electr\'onicos de uniformes y de calidad se ha activado la
    generaci\'on de hiperv\'inculos en el documento usando el paquete \texttt{hyperref}.
    De esta forma el fichero pdf resultante contiene un \'indice adem\'as de
    posibilidad de navegar por el documento como en una p\'agina web al
    ser todas las citas, referencias cruzadas, notas al pi\'e de p\'agina y urls
    hiperenlaces. Adem\'as, el contenido de los distintos \'indices tambi\'en
    son hiperenlaces.
\end{description}

En las siguientes secciones est\'an dedicadas a hacer un resumen de las
posibilidades de esta plantilla (y de \hologo{LaTeX} en general).

\section{Entornos enumerados}

Existen tres entornos enumerados que se pueden anidar de cualquiera de las
maneras posibles. A continuaci\'on hacemos una ilustraci\'on de ellos.

\subsection{Listas}
\begin{itemize}
  \item Soy un elemento de la lista.
  \item Soy otro elemento de la lista.
  \begin{itemize}
    \item Soy un elemento anidado.
    \item Soy otro elemento anidado.
  \end{itemize}
  \item Soy el \'ultimo elemento de la lista.
\end{itemize}

\subsection{Enumeraciones}
\begin{enumerate}
  \item Soy el primer elemento de la enumeraci\'on.
  \begin{enumerate}
    \item Soy un elemento anidado de una enumeraci\'on.
    \item Soy otro elemento anidado como enumeraci\'on.
  \end{enumerate}
  \item Soy el segundo elemento de la enumeraci\'on.
  \begin{itemize}
    \item Soy un elemento anidado de una lista.
    \item Soy otro elemento anidado como lista.
  \end{itemize}
  \item Soy el \'ultimo elemento enumeraci\'on.
\end{enumerate}

\subsection{Descripciones}

En el cap\'itulo~\ref{cha:introduccion} podemos ver un ejemplo de una
descripci\'on. Ponemos aqu\'i otro por completitud.
\begin{description}
  \item[Corta:] Esta es una etiqueta peque\~na con con algo de texto que habla
    sobre ella misma. El texto se tiene la consistencia de un p\'arrafo con
    sangr\'ia francesa.
  \item[Etiqueta bastante larga:] Esta es una etiqueta m\'as larga. N\'otese
    que el texto se sangra como en la etiqueta anterior y no a la altura
    de esta etiqueta.
  \item[Otra etiqueta:] Esta etiqueta esta puesta para ver como quedan los
    textos que tiene m\'as de un p\'arrafo. Primero ponemos un poco de texto
    de relleno.

    Ahora empezamos otro p\'arrafo para que se vea como se respeta la distancia
    entre p\'arrafos y la sangr\'ia contin\'ua siendo la misma.
\end{description}

\section{Entornos flotantes}

Una de las grandes ventajas de \hologo{LaTeX}e\ se encuentra en su algoritmo de
posicionamiento de objetos flotantes. Un objeto flotante es todo aquel que se
le da libertad para moverse dentro del texto hasta ocupar una posici\'on donde
encaje. T\'ipicamente los elementos flotantes suelen ser tablas e im\'agenes
que no puede dividirse entre p\'aginas y por lo tanto deben \enquote{flotar}
hasta su posici\'on. A continuaci\'on presentamos diversos ejemplos de uso.

\subsection{Im\'agenes}

La figura~\ref{fig:ud} es una figura completamente flotante y por eso el
algoritmo la coloca preferentemente al principio de la p\'agina. Sin embargo,
a la figura~\ref{fig:logotipos} se le ha indicado que se prefiere que se coloque
en la posici\'on en la que est\'a (n\'otese el \texttt{[h]}) y por lo tanto
est\'a en medio de la p\'agina. Adem\'as, esta figura est\'a compuesta
por tres subfiguras~\ref{subfig:ud}, \ref{subfig:dt} y \ref{subfig:ESIDE}.

\begin{figure}[!h]
  \centering
  \includegraphics{fig/ud}
  \caption{Logotipo de la Universidad de Deusto.}\label{fig:ud}
\end{figure}

\begin{figure}[h]
  \centering
  \begin{subfigure}{.3\textwidth}
    \centering
    \includegraphics[width=\textwidth]{fig/ud}
    \caption{Logotipo de la Universidad.}\label{subfig:ud}
  \end{subfigure}\quad
  \begin{subfigure}{.3\textwidth}
    \centering
    \includegraphics[width=\textwidth]{fig/dt}
    \caption{Logotipo de DeustoTech.}\label{subfig:dt}
  \end{subfigure}\quad
  \begin{subfigure}{.3\textwidth}
    \centering
    \includegraphics[width=\textwidth]{fig/ESIDE}
    \caption{Logotipo de la Facultad.}\label{subfig:ESIDE}
  \end{subfigure}
  \caption{Ejemplo de subfiguras con los logotipos de la universidad, facultad
           y DeustoTech.}\label{fig:logotipos}
\end{figure}

\subsection{Tablas}

La tabla~\ref{tab:ejemplo} muestra un ejemplo de como se pueden muestran
datos tabulados. N\'otese que al igual que antes es un entorno flotante
y la tabla se desplazar\'a hasta donde encaje. En la tabla~\ref{tab:subtablas}
podemos ver une ejemplo con subtablas que, aunque le hemos indicado que lo
introduzca en esta tabla es demasiado grande y flota hasta la siguiente p\'agina.
Por supuesto, en este caso tambi\'en podemos hacer referencias a las
subtablas~\ref{subtab:ej1} y \ref{subtab:ej2}.

\begin{table}
  \centering
  \caption{Ejemplo de un entorno tabulado.}\label{tab:ejemplo}
  \begin{tabular}{cccc}
    \toprule
      \textbf{Nombre} & \emph{Valor 1} & \emph{Valor 2} & \emph{Valor 3}\\
    \midrule
      Pedro  & 100     & 200     & 300 \\
      Juan   & 200     & 100     & 600 \\
      Mar\'ia& 300     & 50      & 200 \\
      Carmen & 400     & 10      & 7000\\
    \bottomrule
  \end{tabular}
\end{table}

\begin{table}[h]
  \centering
  \caption{Ejemplo de un entorno tabulado con varias subtablas.}\label{tab:subtablas}
  \begin{subtable}{.45\textwidth}\centering
    \begin{tabular}{cccc}
      \toprule
        \textbf{Nombre} & \emph{Valor 1} & \emph{Valor 2} & \emph{Valor 3}\\
      \midrule
        Pedro  & -100     & 200      & $a$ \\
        Juan   & 2000     & -100     & $b$ \\
        Mar\'ia& 300      & 50       & $c$ \\
        Carmen & 400      & -10      & $f$ \\
      \bottomrule
    \end{tabular}
    \caption{Valores de algo.}\label{subtab:ej1}
  \end{subtable}\quad
  \begin{subtable}{.45\textwidth}\centering
    \begin{tabular}{cccc}
      \toprule
        \textbf{Nombre} & \emph{Valor 1} & \emph{Valor 2} & \emph{Valor 3}\\
      \midrule
        Pedro  & 100     & $\alpha$& 300 \\
        Juan   & $\gamma$& 100     & 600 \\
        Mar\'ia& 300     & 50      & $\beta$ \\
        Carmen & $\omega$& 10      & $\varepsilon$\\
      \bottomrule
    \end{tabular}
    \caption{Valores de otra cosa.}\label{subtab:ej2}
  \end{subtable}
\end{table}

\subsection{Algoritmos}

Finalmente terminamos con el \'ultimo entorno flotante que falta: los algoritmos.
Este entorno flotante es el m\'as extra\~no pues necesita interpretar
de forma completamente distinta el texto. Los algoritmos~\ref{lst:holamundo} y
\ref{lst:python} hay dos ejemplos muy simples de sus posibilidades.

\begin{listing}\centering
  \begin{minipage}{.4\textwidth}
    \begin{minted}[linenos=true]{c}
int main() {
  printf("hello world");
  return 0;
}
    \end{minted}
  \end{minipage}
  \caption{\enquote{Hola mundo} en C.}\label{lst:holamundo}
\end{listing}

\begin{listing}\centering
  \begin{minipage}{.4\textwidth}
    \begin{minted}[linenos=true,mathescape,gobble=6]{python}
      # Returns $\sum_{i=1}^{n}i$
      def sum_from_one_to(n):
        r = range(1, n + 1)
        return sum(r)
    \end{minted}
  \end{minipage}
  \caption{Ejemplo m\'as elaborado en Python.}\label{lst:python}
\end{listing}

\section{Referencias cruzadas y citas bibliogr\'aficas}

Una de las grandes caracter\'isticas de \hologo{LaTeX} es la facilidad para
hacer tanto referencias cruzadas como citas bibliogr\'aficas. Como ya se ha
visto anteriormente, para referenciar un cap\'itulo, secci\'on, etc. o
un elemento flotante solo es necesario asignarle un \verb+\label{}+ y
posteriormente citarle con \verb+\ref{}+. Por ejemplo, para referenciar
el primer cap\'itulo basta con colocar la etiqueta
\verb+\label{cha:introduccion}+ en cualquier punto del cap\'itulo
(preferiblemente despu\'es del t\'itulo) y luego hacer referencia a \'el
con \verb+\ref{cha:introduccion}+. Es decir, hacemos referencia al
cap\'itulo~\ref{cha:introduccion}. De la misma forma podemos hacer referencia a
los elementos flotantes como figuras, tablas y algoritmos.

Finalmente, la bibliograf\'ia se genera de forma autom\'atica conteniendo
\'unicamente los trabajo expl\'icitamente citados en la
memoria\footnote{Hay varios mecanismos para incluir tambi\'en otras obras.
V\'ease la documentaci\'on de \textsc{Bib}\hologo{LaTeX}.}. Para ello es necesario
contar una base de datos bibliogr\'afica en un formato reconocido. El m\'as
tradicional es el formato \texttt{.bib}. El fichero \texttt{bib.bib}
incluido en esta memoria es un ejemplo b\'asico con el que se han construido
las citas de este documento.

A modo de ejemplo ponemos aqu\'i una lista de citas de relleno para completar
la bibliograf\'ia \cite{article,conference,book}, \cite{phd} y \cite{master}.

\printbibliography[heading=bibintoc]

\appendix

\chapter{Normativa}\label{an:normativa}

En este ap\'endice mostramos como dividir textos grandes en ficheros m\'as
manejables mediante el comando \verb+\input{}+. Para ello incluiremos la
normativa en este anexo.

\section{Introducci\'on}

En este documento se recogen las especificaciones más relevantes relacionadas con el formato de
las memorias de los proyectos de la asignatura \textbf{Proyecto Fin de Grado} del Grado en Ingeniería
Informática, Grado en Ingeniería en Tecnologías Industriales, Grado en Ingeniería en Electrónica
Industrial y Automática, Grado en Ingeniería en Organización Industrial y Grado en Ingeniería
en Tecnologías de Telecomunicación.

\section{Encuadernaci\'on}

La encuadernación será en cartulina (con cola y cinta negra), del color especificado a continuación
según titulación (las tonalidades específicas están disponibles en la cartelera).

\begin{center}
  \begin{tabular}{ll}
    \toprule
      \textbf{Titulaci\'on} & \textbf{Color de la encuadernación}\\
    \midrule
      Grado en Ingeniería Informática                             &  Azul mediterráneo \\
      Grado en Ingeniería en Organización Industrial              &  Crema             \\
      Grado en Ingeniería en Electrónica Industrial y Automática  &  Rojo Navidad      \\
      Grado en Ingeniería en Tecnologías de Telecomunicación      &  Verde Hierba      \\
      Grado en Ingeniería en Tecnologías Industriales             &  Butano            \\
    \bottomrule
  \end{tabular}
\end{center}

\section{Portada}

La portada consta de un encabezado que respeta la identidad corporativa de la Universidad de
Deusto y que se facilita a los estudiantes a través de la página web de la asignatura. En dicho
encabezado figuran en castellano y en euskera los datos relativos a la Universidad, la Facultad y la
titulación correspondiente.

\textbf{Dada la complejidad de definir literalmente toda la distribución de los diversos contenidos de la
portada, se facilitan plantillas en formato \enquote{pdf editable} por cada titulación y \emph{que son de
obligado uso}.}

El orden de los párrafos será:
\begin{itemize}
  \item La expresión: \textbf{Proyecto fin de grado}.
  \item El título completo (sólo se escribirán con mayúscula la primera letra de la primera palabra
        y los nombres propios; no se pone punto al final)
  \item El nombre del estudiante
  \item El director o directora, donde se especificará dicho cargo, tal como figura en el ejemplo, es
        decir: \textbf{Director}: o \textbf{Directora}:. No se debe especificar el título de éste (doctor, licenciado o
        ingeniero)
  \item  La fecha, tal como figura en el ejemplo. \textbf{Bilbao, mes de año} (en número)
\end{itemize}

\section{Papel}

Ha de emplearse papel del mismo color (blanco), calidad 80 gr. mínimo y normalmente suficiente y
tamaño (UNE A-4), en todas las copias de la memoria.

\section{Impresi\'on}

\textbf{Se empleará la plantilla \enquote{PlantillaPFG}} disponible en la página web de la asignatura (en la sección
de Plantillas) con el fin de respetar las fuentes, los márgenes y los encabezados y pies de páginas.
Se suministra un ejemplo de su uso en formato Word y \hologo{LaTeX}.

\textbf{El uso de estas plantillas (y no otras) es de obligado cumplimiento} para lograr una uniformidad en
todas las titulaciones y memorias. Se deberán mantener el formato de las fuentes, títulos y
márgenes establecidos en estos documentos.

Los márgenes han de ajustarse a las siguientes medidas: superior e inferior 3 cm, lateral interno 3’5
cm y lateral externo 2’5 cm.

El encabezado de las \textbf{páginas pares indicará el título del capítulo en curso} y el encabezado de las
\textbf{páginas impares llevará los términos \enquote{PROYECTO FIN DE GRADO}}, según plantilla. La numeración
de las páginas será exterior. Las páginas en blanco no deben llevar numeración ni encabezado y las
páginas posteriores deben seguir la secuencia de la numeración del documento.

Si hay alguna imagen, tabla o gráfico que deban ir en horizontal, se debe rotar la imagen, no la
página completa. El encabezado y pie de página debe ser los mismos en toda la memoria, a
excepción de los anexos, que pueden llevar un formato distinto, pero debe estar indicado
claramente que se trata de anexos.

La presentación de la memoria debe cuidarse con especial esmero, procurando claridad, limpieza,
uniformidad y ausencia de erratas. Se imprimirá a doble cara. Se evitarán los saltos de página
innecesarios y la presencia de páginas que por una u otra razón estén prácticamente en blanco.

Las tres copias de la memoria que se entregan deben ser idénticas y la impresión se puede realizar
en B/N o color.

\section{Organización del contenido}
\subsection{Primera página}
Debe ser exactamente igual que la portada, pero en papel blanco de la misma calidad que el
resto de la memoria.

Esta página es la que \textbf{deberá llevar a modo de confirmación la firma del director/a del
proyecto. Esta firma deberá estar encima del nombre del director/a.}

\subsection{Segunda página}
La segunda página es el reverso de la primera página, irá en blanco (sin número ni encabezado).

\subsection{Resumen y descriptores}
En la tercera página (anverso de la segunda hoja) debe aparecer un resumen del proyecto (de
200 a 250 palabras) y a continuación del mismo, entre tres y cinco descriptores (palabras clave)
que ayuden a clasificar adecuadamente el proyecto.

La paginación anterior al capítulo 1 (portada, resumen e índice) se hará con números romanos
(i, ii, iii, iv,\dots). En la primera página, la portada, no debe aparecer el número.

\subsection{\'Indice}
El índice empezará en la página 5 (v), es decir, en el anverso de la tercera hoja.

\subsection{Memoria}
El capítulo 1 comenzará en la página 1. El resto de capítulos \textbf{deberán comenzar también en
página impar}.

\subsection{Bibliograf\'ia}

Este será el último capítulo del documento e irá antes de los anexos. El formato de la
bibliografía es el indicado en los ejemplos de memoria y plantillas disponibles, tanto en \hologo{LaTeX}
como en Word. De modo general, el formato a emplear será:

\begin{description}
  \item [Para referenciar páginas web:] \enquote{Titulo/Nombre de la Web},
        \url{http://www.google.es}, (consultado el 7/10/12).
  \item [Para referenciar libros:] Autores separados por comas, \enquote{Título de la contribución/Libro},
        \emph{Editorial en cursiva}, Año de la publicación.
\end{description}

\subsection{Planos}
Si los hubiera, la presentación de planos se hará de acuerdo a la normativa vigente (UNE-EN ISO
7200:2004) en esta área.

\subsection{Anexos}
Los anexos irán adjuntados al final de la memoria y podrán tener un formato libre diferente al
del resto de la memoria.

\subsection{Otros}
No se establece ninguna otra restricción en cuanto al contenido y/o formato de las memorias, a
excepción de las que se recogen en el presente documento.



\backmatter

\end{document}
